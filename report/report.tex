\documentclass[a4paper, 11pt]{article}
\usepackage[english]{babel}
\usepackage[T1]{fontenc}
\usepackage[utf8]{inputenc}

% Import images from media folder.
\usepackage{graphicx}
\graphicspath{ {./media/} }

% Header & Footer
% \usepackage{fancyhdr}
% \pagestyle{fancy}
% \lhead{Konrad M. L. Claesson}

\usepackage{soul, color} % Strikethrough, highlight
\usepackage{enumitem} % Enumerate using letters, a) b) c), ...

% No default paragraph indentation.
\setlength{\parindent}{0px}
\renewcommand{\baselinestretch}{1.325}

% Change Paragraph Spacing
\usepackage{setspace}

% Double quotes, \say{}.
\usepackage{dirtytalk}

% Block quotes
\usepackage{etoolbox}
\AtBeginEnvironment{quote}{\singlespacing\small}

% Links
\usepackage{hyperref}
\definecolor{urlblue}{RGB}{000, 106, 219}
\hypersetup{
   colorlinks=true,  % false: boxed links; true: colored links
   linkcolor=black,  % color of internal links
   urlcolor=urlblue
}

% 1st, 2nd, ...
\usepackage{nth}

% Tables
\usepackage{longtable}

% Proofs
\usepackage{logicproof}

% Code
\usepackage{fancyvrb}
\usepackage{listings}

\lstset
{
  basicstyle=\small\ttfamily,
  breaklines=true,
  columns=fullflexible
}

% Cover Page
\title{Prolog-Based Natural Deduction Proof Verification
Algorithm}
\author
{
   Konrad M. L. Claesson \\
   konradcl@kth.se \\
   Lab 2 $-$ DD1351
}
\date{\today}

\begin{document}
   \maketitle
   \thispagestyle{empty}   % remove page number
   
   % Restart numbering
   \clearpage
   \pagenumbering{arabic}
   \newpage

   \tableofcontents
   \newpage

   \section{Introduction}
   Natural deduction is a proof calculus comprised of a set of
   inference rules that are used to infer formulas from other
   formulas. The rules can be categorized into introduction
   and elimination rules. The former combine or negate
   formulas by introducing connectives; the latter decompose
   them by eliminating connectives. These rules are defined in
   terms of formula patterns that must hold true for a 
   connective to be introduced or eliminated. For instance, to
   claim $\phi \wedge \psi$, it must be known that
   both of the formulas $\phi$ and $\psi$ are true. Prolog is
   a logic programming language that enables easy expression 
   of logical statements. To exemplify, the Prolog code
   
\begin{verbatim}
   rule(_, [_, and(X, Y), andint(R1, R2)], Verified) :-
      member([R1, X, _], Verified),
      member([R2, Y, _], Verified).
\end{verbatim}
   
   states that $\phi \wedge \psi$ can be concluded, by the and
   introduction rule, if both $\phi$ and $\psi$ are known to
   be true. 
   \bigbreak

   This report presents a Prolog-based algorithm that can
   verify the validity of any proof in propositional logical 
   that has been written exclusively using the rules of
   natural deduction listed in the below table.

   \begin{table}[h]
      \centering
      \includegraphics[scale=0.325]{inference-rules}
      \caption{Algorithm-Supported Proof Rules}
      \label{algorithm-supported-proof-rules}
   \end{table}

   \section{Method}
   
   A proof is valid if all of its line are valid. A line is
   valid if it can be inferred by applying the rules of
   natural deduction to the proof's premises and previously
   inferred formulas. Accordingly, the validity of a proof can
   be determined if, and only if, the premises and previously
   derived formulas are known, and the required conditions for
   the employment of all used rules are defined. To this end,
   19 instances of the \texttt{rule/3} predicate were defined,
   one for each rule from Table 
   \ref{algorithm-supported-proof-rules}, that given the
   premises, a line, and all previous lines, returns 
   \texttt{true} if the passed in line can be inferred from
   the premises and previous lines, and \texttt{false}
   otherwise. The set of all \texttt{rule/3} predicates 
   achieve this behavior by, first, matching the rule used to
   deduce the passed in line (which is specified at the end 
   of the line) with the predicate handling that rule; and
   secondly, verifying that the premises and derived formulas 
   that were used to invoke the rule satisfy the conditions 
   for its invocation. The premises of a proof are specified
   in the input file to the algorithm and stored in a list
   called \texttt{Premises}. Verified lines are appended to a 
   list, usually named \texttt{Verified}, immediately after a
   \texttt{rule/3} predicate has declared them as
   \texttt{true}.
   \bigbreak

   Naturally, handling of the algorithm-supported rules breaks
   down into premise handling, handling of inference rules,
   and assumption handling. How each are handled is
   described in sections 
   \ref{premise-handling},
   \ref{handling-of-inference-rules}, and
   \ref{assumption-handling}, respectively. 
   Thereafter, two examples of how the algorithm verifies
   proofs are given.

   \subsection{Input Handling}
   \label{input-handling}
   
   Every input file is comprised of three Prolog terms:
   \begin{enumerate}
      \item A list of premises (left-hand side of sequent).
      \item A conclusion (right-hand side of sequent).
      \item A proof in natural deduction.
   \end{enumerate}

   For example, the sequent 
   $\neg (p \vee q) \vdash \neg p \wedge \neg q$ with the
   proof

   \begin{logicproof}{2}
      \neg (p \lor q) & premise \\
      \begin{subproof}
         p & assumption \\
         p \lor q & $\lor\mathrm{i}_1$ 2 \\
         \perp & $\neg$e 3, 1
      \end{subproof}
      \neg p & $\neg$i 2-4 \\
      \begin{subproof}
         q & assumption \\
         p \lor q & $\lor\mathrm{i}_2$ 6 \\
         \perp & $\neg$e 7, 1
      \end{subproof}
      \neg q & $\neg\mathrm{i}$ 6-8 \\
      \neg p \land \neg q & $\land\mathrm{i}$ 5, 9
   \label{de-morgan-proof}
   \end{logicproof}

   is entered into the algorithm in the below format.
   \newpage

\begin{verbatim}
   % Premises
   [neg(or(p, q))].

   % Conclusion
   and(neg(p), neg(q)).
   
   % Proof
   [
      [1,      neg(or(p, q)),          premise],
      [
         [2,   p,                      assumption],
         [3,   or(p, q),               orint1(2)],
         [4,   cont,                   negel(3, 1)]
      ],
      [5,      neg(p),                 negint(2, 4)],
      [
         [6,   q,                      assumption],
         [7,   or(p, q),               orint2(6)],
         [8,   cont,                   negel(7, 1)]
      ],
      [9,      neg(q),                 negint(6, 8)],
      [10,     and(neg(p), neg(q)),    andint(5, 9)]
   ].
\end{verbatim}

   In this format, the premises, conclusion and proof can be
   read into \texttt{Premises}, \texttt{Conclusion} and
   \texttt{Proof}, respectively, and fed into the proof
   verification algorithm \texttt{verify\_proof/3}, via the
   following code.

\begin{verbatim}
   verify(InputFileName) :-
     see(InputFileName),
     read(Premises), read(Conclusion), read(Proof),
     seen,
     valid_proof(Premises, Conclusion, Proof).
\end{verbatim}

   \subsection{The Recursive Helper Predicates}
   \label{recursive-helper-predicates}

   When the input file has been read, the verification
   algorithm is initiated by a call to
   \texttt{valid\_proof/3}. 
   
\begin{verbatim}
   valid_proof(Premises, Conclusion, Proof) :- 
      verify_end(Conclusion, Proof),
      verify_proof(Premises, Proof, []).
\end{verbatim}

   This helper predicate takes in the
   \texttt{Premises}, \texttt{Proof} and \texttt{Conclusion}
   from the input file and invokes the 
   \texttt{verify\_end(Conclusion, Proof)} and \\
   \texttt{verify\_proof(Premises, Proof, [])} predicates.
   \bigbreak

   \texttt{verify\_end(Conclusion, Proof)} ensures 
   that the final line of the proof is equal to the
   sequent's conclusion. 
   
\begin{verbatim}
   verify_end(Conclusion, Proof) :-
       last(Proof, Line),              
       nth0(1, Line, ProofConclusion),
       Conclusion = ProofConclusion.
\end{verbatim}
   
   This is achieved by
   first applying Prolog's built-in \texttt{last/3}
   predicate to extract the last element (line) in
   \texttt{Proof}, and then utilizing \texttt{nth0/3} 
   to select the formula present on the last line. If
   the proof is valid, this formula is equal to
   the sequent's conclusion. Accordingly, 
   \texttt{verify\_end} concludes with an equality
   check that ensures that the described equality
   holds.
   \bigbreak

   \texttt{verify\_proof(Premises, Proof, Verified)}
   recursively validates the provided proof
   line-by-line, from top to bottom. 
   
\begin{verbatim}
   verify_proof(_, [], _) :- !.
   verify_proof(Premises, [H|T], Verified) :-
      rule(Premises, H, Verified),
      append(Verified, [H], VerifiedNew),
      verify_proof(Premises, T, VerifiedNew).
\end{verbatim}

   It validates each 
   line (element in \texttt{Proof}) by invoking 
   \texttt{rule/3} with the premises, line, and 
   previously verified lines as arguments. 
   \texttt{rule/3} then returns \texttt{true} whenever 
   the line follows by natural deduction from the 
   premises and previously verified lines, and false
   otherwise. If a line is valid, it gets appended to
   the \texttt{Verified} list and \texttt{verify\_proof}
   is called again with the yet unverified lines and
   the new list of verified lines.
   \bigbreak

   The subsequent table summarizes the conditions under which
   the helper predicates return true.

\begin{table}[ht]
\centering
\begin{tabular}{|l|l|l|}
   \hline
      \textbf{Predicate} 
      & \textbf{Parameters}                                  
      & \textbf{Truth Conditions} \\
   \hline
      \texttt{verify}             
      & \texttt{InputFileName}  
      & \texttt{valid\_proof} is true \\
   \hline
      \texttt{valid\_proof}       
      & \begin{tabular}[c]{@{}l@{}}
         \texttt{Premise} \\ 
         \texttt{Conclusion} \\ 
         \texttt{Proof}
      \end{tabular}          
      & \begin{tabular}[c]{@{}l@{}}
         \texttt{verify\_end} is true, and \\
         \texttt{verify\_proof} is true
      \end{tabular} \\  
   \hline
      \texttt{verify\_end}        
      & \begin{tabular}[c]{@{}l@{}}
         \texttt{Conclusion} \\ 
         \texttt{Proof}
      \end{tabular}                    
      & Proof ends with sequent's conclusion. \\
   \hline
      \texttt{verify\_proof}      
      & \begin{tabular}[c]{@{}l@{}}
         \texttt{Premises} \\ 
         \texttt{Proof} \\ 
         \texttt{Verified}
      \end{tabular}           
      & \begin{tabular}[c]{@{}l@{}}
         The first line, and recursively all \\ 
         subsequent lines, are provable by \\ 
         natural deduction.
      \end{tabular} \\ 
   \hline
\end{tabular}
\caption{Truth Conditions of the Helper Predicates}
\end{table}

   \subsection{The Rule Predicate}
   \texttt{rule/3} takes in the list of premises, a
   line or assumption box, and the, as of then, 
   verified lines of a proof. It then ensures that
   the line or assumption box follows by natural
   deduction from the premises and previous lines of
   the proof. It does this by inspecting the last
   element of each line, which specifies the
   proof rule that was applied to deduce the
   formula on that line, and then checking that the
   conditions for using the rule are fulfilled.
   \bigbreak

   In the subsections ahead the various cases of
   \texttt{rule/3} are explored in greater detail.

   \subsubsection{Premise Handling}
   \label{premise-handling}
   Lines that claim to be premises are verified by checking
   that the \texttt{Formula} on the given line is a member of 
   the proof's \texttt{Premises}. This is done by the below
   code, 
   
\begin{verbatim}
   rule(Premises, [_, Formula, premise], _) :-
      member(Formula, Premises).
\end{verbatim}

   where the \texttt{member/2} predicate resolves
   whether \texttt{Formula} is a member of \texttt{Premiss}. 

   \subsubsection{Handling of Inference Rules}
   \label{handling-of-inference-rules}

   As described earlier, a line gets matched with a 
   \texttt{rule/3} predicate based on the inference rule that 
   was used to derive it. The following table states under 
   what conditions each of the 17 inference rules evaluate to
   \texttt{true}. An implementation of each rule can be found
   in the algorithm source code in the appendix.
   \bigbreak


   \begin{longtable}[ht]{|l|l|l|}
      \hline
         \textbf{Logical Rule} 
         & \textbf{Truth Conditions} \\
      \hline
         \texttt{copy(x)} 
         & \begin{tabular}[c]{@{}l@{}}
            For lines that are deduced by the \\
            \textit{copy} rule, the copied formula must \\
            appear previously in the proof \\
            (i.e. be a member of \texttt{Verified}).
         \end{tabular} \\ 
      \hline
         \texttt{negnegint(x)} 
         & \begin{tabular}[c]{@{}l@{}}
            For lines with a formula $\neg \neg \phi$, that 
            is \\
            deduced by \textit{double negation}, \\
            \textit{introduction}, the formula $\phi$ must \\
            appear previously in the proof (i.e. \\
            exist in \texttt{Verified}). 
         \end{tabular} \\
      \hline
         \texttt{negnegel(x)} 
         & \begin{tabular}[c]{@{}l@{}}
            For lines with a formula $\phi$ that is \\
            deduced by \textit{double negation} \\ 
            \textit{elimination}, the formula $\neg \neg \phi$
            must \\
            appear previously in the proof (i.e \\
            exist in \texttt{Verified}).
         \end{tabular} \\
      \hline
         \texttt{andint(x,y)} 
         & \begin{tabular}[c]{@{}l@{}}
            For a line with a formula $\phi \wedge \psi$ \\
            that is deduced by \textit{and introduction}, 
            the \\
            formulas $\phi$ and $\psi$ must appear \\
            previously in the proof (i.e. exist in \\
            \texttt{Verified}).
         \end{tabular} \\
      \hline
         \texttt{andel1(x)} 
         & \begin{tabular}[c]{@{}l@{}}
            For a line with a formula $\phi$, that is \\
            deduced by \nth{1} \textit{and elimination}, a \\
            formula of the form $\phi \wedge \psi$ must \\
            appear previously in the proof (i.e. \\
            exist in \texttt{Verified}).
         \end{tabular} \\
      \hline
         \texttt{andel2(x)} 
         & \begin{tabular}[c]{@{}l@{}}
            For a line with a formula $\phi$, that is \\
            deduced by \nth{2} \textit{and elimination}, \\
            a formula of the form $\psi \wedge \phi$ must \\
            appear previously in the proof (i.e. \\
            exist in \texttt{Verified}).
         \end{tabular} \\
      \hline
         \texttt{orint1(x)} 
         & \begin{tabular}[c]{@{}l@{}}
            For a line with a formula $\phi \vee \psi$, 
            that \\
            is deduced by \nth{1} \textit{or introduction}, \\
            a formula $\phi$ must appear previously \\
            in the proof (i.e. exist in \texttt{Verified}).
         \end{tabular} \\
      \hline
         \texttt{orint2(x)} 
         & \begin{tabular}[c]{@{}l@{}}
            For a line with a formula $\phi \vee \psi$, 
            that \\
            is deduced by \nth{2} \textit{or introduction}, \\
            a formula $\psi$ must appear previously \\
            in the proof (i.e. exist in \texttt{Verified}).
         \end{tabular} \\
      \hline
         \texttt{orel(x,y,u,v,w)} 
         & \begin{tabular}[c]{@{}l@{}}
            For a line with a formula $\chi$, that is \\
            deduced by \textit{or elimination}, a formula \\
            $\phi \vee \psi$, a box assuming $\phi$ and 
            ending \\
            in $\chi$, and a box assuming $\psi$ and \\
            ending in $\chi$, must appear previously \\
            in the proof (i.e. exist in \texttt{Verified}).
         \end{tabular} \\
      \hline
         \texttt{impint(x,y)} 
         & \begin{tabular}[c]{@{}l@{}}
            For a line with a formula 
            $\phi \rightarrow \psi$, \\
            that is deduced by \textit{implication} \\
            \textit{introduction}, a box assuming 
            $\phi$ and \\
            concluding in $\psi$ must appear \\
            previously in the proof (i.e. exist in \\
            \texttt{Verified})
         \end{tabular} \\
      \hline
         \texttt{impel(x,y)} 
         & \begin{tabular}[c]{@{}l@{}}
         For a line with a formula $\psi$ that is \\
         deduced by \textit{implication elimination} \\
         a formula $\phi$, and a box assuming $\phi$ \\
         and ending in $\psi$, must appear \\
         previously in the proof (i.e. exist in \\
         \texttt{Verified}).
         \end{tabular} \\
      \hline
         \texttt{negint(x,y)} 
         & \begin{tabular}[c]{@{}l@{}}
         For a line with a formula $\neg \phi$ that is \\
         deduced by \textit{negation introduction} a \\
         box assuming $\phi$ and concluding in $\perp$ \\
         must appear previously in the proof \\
         (i.e. exist in \texttt{Verified}).
         \end{tabular} \\
      \hline
         \texttt{negel(x,y)} 
         & \begin{tabular}[c]{@{}l@{}}
         For a line with a contradiction $\perp$ \\
         that is deduced by \textit{negation} \\
         \textit{elimination} two formulas $\phi$ and 
         $\neg \phi$ \\
         must appear previously in the proof \\
         (i.e. exist in \texttt{Verified}).
         \end{tabular} \\
      \hline
         \texttt{contel(x)} 
         & \begin{tabular}[c]{@{}l@{}}
         For a line with a formula $\phi$ that is \\
         deduced by \textit{contradiction} \\
         \textit{elimination} a line with a \\
         contradiction $\perp$ must appear \\
         previously in the proof (i.e. exist \\
         in \texttt{Verified}).
         \end{tabular} \\
      \hline
         \texttt{mt(x,y)} 
         & \begin{tabular}[c]{@{}l@{}}
         For a line with a formula $\neg \phi$ that is \\
         deduced by \textit{modus tollens} a formula \\
         $\phi \rightarrow \psi$, and a formula $\neg \psi$,
         must \\
         appear previously in the proof (i.e. \\
         exist in \texttt{Verified}).
         \end{tabular} \\
      \hline
         \texttt{pbc(x,y)} 
         & \begin{tabular}[c]{@{}l@{}}
         For a line with a formula $\phi$ that is \\
         derived by a \textit{proof by contradiction} \\
         a box assuming $\neg \phi$ and ending in a \\
         contradiction $\perp$ must appear \\
         previously in the proof (i.e. exist in \\
         \texttt{Verified}).
         \end{tabular} \\
      \hline
         \texttt{lem} 
         & \begin{tabular}[c]{@{}l@{}}
         By the \textit{law of the excluded middle} a \\
         line with a formula of the form 
         $\phi \vee \neg \phi$ \\
         is always valid.
         \end{tabular} \\
      \hline
   \caption{Truth Conditions for Inference Rules}
   \end{longtable}

   \subsubsection{Assumption Handling}
   \label{assumption-handling}
   
   The most outstanding \texttt{rule/3} predicate is the one
   that handles assumptions. There are two reasons for this.
   Firstly, rather than being matched with lines that declare
   themselves as \say{derived by assumption}, it is matched to
   boxes (implemented as a list of lines) that start with an 
   assumption. Secondly, unlike the other predicates, it does
   not check for the fulfillment of a set of predefined
   conditions; instead, it treats the lines of the box as a
   proof in its own right, where the assumption becomes a
   premise, and returns \texttt{true} only if the proof within
   the box is valid. In other words, the algorithm handles
   assumptions by regarding their associated boxes as 
   sub-proofs that can be verified recursively.
   \bigbreak

   The subsequent code handles assumptions in the described
   manner,

\begin{verbatim}
rule(Premises, [[R, Formula, assumption] | T], Verified) :-
   append(Verified, [[R, Formula, assumption]], VerifiedNew),
   verify_proof(Premises, T, VerifiedNew).
\end{verbatim}
   
   where \texttt{Premises} is the list of premises supplied 
   in the input file, \\ 
   \texttt{[[R, Formula, assumption] | T]} is the assumption
   box, and \texttt{Verified} is the list of verified lines.
   The \texttt{append/3} predicate appends the assumption line
   to the list of verified lines. Then, \texttt{verify\_proof}
   is recursively called to verify the sub-proof within the
   box.

   \subsection{Example of Proof Verification}

   In this subsection, two examples of how the proof
   verification algorithm works are provided. First, a
   detailed explanation of how the algorithm progresses from
   accepting inputs to concluding the validity of a coherent
   proof is given. Then, a shorter description of the 
   algorithm's behavior in the event of an invalid proof is 
   presented.
   
   \subsubsection{Verification of a Valid Proof}

   The proof of the sequent 
   $\neg(p \lor q) \vdash \neg p \land \neg q$ provided on
   page \pageref{de-morgan-proof} is valid. Appropriately, 
   the proof verification algorithm (view appendix) also
   identifies this as a valid proof. To do this, the 
   algorithm starts out by calling \\
   \texttt{verify(InputFileName)}, which reads in the proof 
   from the specified file (see Section \ref{input-handling}).
   Then, \texttt{valid\_proof(Premises, Conclusion, Proof)} 
   is invoked, which in turn calls
   \texttt{verify\_end(Conclusion, Proof)} and \\
   \texttt{verify\_proof(Premises, Proof, [])}, in the given 
   order (see Section \ref{recursive-helper-predicates}).
   The \texttt{verify\_proof/3} predicate recursively
   verifies each line by invoking \texttt{rule/3} and
   calling itself. In the above proof, the first
   line invokes the premise verification rule

\begin{verbatim}
   rule(Premises, [_, Formula, premise], _) :-
      member(Formula, Premises).
\end{verbatim}
                    
   which checks that \texttt{Formula} (the formula
   on the given line) is a member of
   \texttt{Premises}. If \texttt{Formula} is a
   member, the line is valid and \texttt{rule} returns
   \texttt{true}. Line two is an assumption and thus 
   opens a box. It is handled by the assumption rule

\begin{verbatim}
rule(Premises, [[R, Formula, assumption] | T], Verified) :-
   append(Verified, [[R, Formula, assumption]], VerifiedNew),
   verify_proof(Premises, T, VerifiedNew).
\end{verbatim}

   which returns \texttt{true} if the sub-proof of the
   assumption box is valid. To verify that the
   assumption box contains a valid proof the
   assumption line is appended to the list of
   verified lines, and \texttt{verify\_proof/3} is
   called to validate the sub-proof. Within the box,
   line three is verified by ensuring that its
   application of the first disjunction introduction
   rule is valid. The below code does this

\begin{verbatim}
   rule(_, [_, or(X, _), orint1(R)], Verified) :-
      member([R, X, _], Verified).
\end{verbatim}

   by confirming that the formula $p$ has been
   deduced earlier in the proof. Line four is deduced
   by administering the rule of negation elimination.
   The below code verifies that the rule can be
   employed

\begin{verbatim}
   rule(_, [_, cont, negel(R1, R2)], Verified) :-
      member([R1, X, _], Verified),
      member([R2, neg(X), _], Verified).
\end{verbatim}

   by checking that both $p \vee q$ and 
   $\neg(p \vee q)$ occur previously in the proof
   (are members of \texttt{Verified}). When the
   sub-proof of the assumption box has been
   validated, the entire box is appended to
   \texttt{Verified} before continuing to verify the
   next line or assumption box. This ensures that no
   single line that is predicated on the assumption
   can be referenced without also referencing the
   assumption. In the above proof, the fifth line, 
   which also is the first line following the
   upper assumption box, is deduced by applying the
   rule of negation introduction. The subsequent code
   verifies that the rule can be applied

\begin{verbatim}
   rule(_, [_, neg(X), negint(R1, R2)], Verified) :-
      member([[R1, X, assumption] | T], Verified),
      last(T, BoxConclusion),
      [R2, cont, _] = BoxConclusion.
\end{verbatim}

   by establishing that the assumption box starts
   with $p$, and that it ends with a contradiction.
   \bigbreak

   Lines six through nine repeat the 
   same deductive pattern as lines two through five.
   The proof's concluding line is arrived at by
   utilizing the rule of conjunction introduction.
   The employment of this rule is validated by the
   below code

\begin{verbatim}
   rule(_, [_, and(X, Y), andint(R1, R2)], Verified) :-
      member([R1, X, _], Verified),
      member([R2, Y, _], Verified).
\end{verbatim}

   which states that the rule can be administered if,
   and only if, both $\neg p$ and $\neg q$ occur
   previously in the proof (exist in
   \texttt{Verified}).

   \subsubsection{Verification of an Invalid Proof}
   
   The following is an invalid proof of the same sequent as
   above. Lines one through three are deduced by rules that 
   have already been addressed, and hence they will not be 
   discussed any further. Nevertheless, the fourth and final 
   line incorrectly applies the rule of negation introduction.
   The subsequent text examines how the proof verification 
   algorithm handles this.

   \begin{logicproof}{2}
      \neg (p \lor q) & premise \\
      \begin{subproof}
         p \lor q & assumption \\
         \perp & $\neg$e 2, 1
      \end{subproof}
      \neg p \land \neg q & $\neg$i 2-3 
   \label{de-morgan-proof-invalid}
   \end{logicproof}
   \bigbreak

   \begin{figure}[h]
   \centering
\begin{BVerbatim}
[1,      neg(or(p, q)),       premise],
[
   [2,   or(p, q),            assumption],
   [3,   cont,                negel(2, 1)]
],
[4,      and(neg(p), neg(q))  negint(2, 3)]
\end{BVerbatim}
   \end{figure}
   \bigbreak

   Recall that the negation introduction rule can be employed
   to conclude $\neg \phi$, for some formula $\phi$, if, and 
   only if, there exists a box starts with an assumption of
   $\phi$ and ends in a contradiction. The proof passes the
   latter requirement, which is enforced by the code

\begin{verbatim}
   last(T, BoxConclusion),
   [R2, cont, _] = BoxConclusion.
\end{verbatim}

   but fails the former, as the assumption $\phi = p \vee q$
   combined with the contradiction and negation introduction 
   rule, can only be used to conclude $\neg (p \vee q)$, and
   not $\neg p \vee \neg q$. Hence, the called
   \texttt{rule/3} predicate fails. This failure propagates
   through \texttt{verify\_proof/3} and 
   \texttt{valid\_proof/3} all the way to \texttt{verify/1},
   where it ultimately causes the proof to be declared as
   invalid.

   \section{Results}
   The algorithm correctly identified the validity of every
   minimal, rule-specific, proof. It also accurately verified
   all multi-step proofs.
   \bigbreak

   The algorithm along with the proofs in an input-ready
   format can be found by following this link: 
   \url{github.com/konradcl/prolog-proof-verifier}. The
   algorithm has also been appended to this report.

   \newpage
   \section*{Appendix}
   
\begin{lstlisting}
:- discontiguous(rule/3).

verify(InputFileName) :-
  see(InputFileName),
  read(Premises), read(Conclusion), read(Proof),
  seen,
  valid_proof(Premises, Conclusion, Proof).

% The proof is valid if the proof ends with the sequent's
% conclusion.
valid_proof(Premises, Conclusion, Proof) :- 
   verify_end(Conclusion, Proof),
   verify_proof(Premises, Proof, []).

% Verifies that proof ends with the sequent's conclusion.
verify_end(Conclusion, Proof) :-
    last(Proof, Line),              % line := last proof line
    nth0(1, Line, ProofConclusion), % select ProofConclusion
    Conclusion = ProofConclusion.

% Recursively verifies the proof line by line.
%
% [H|T] = Proof
% Verified:    previously verified proof lines
% VerifiedNew: previously and newly verified proof lines
verify_proof(_, [], _) :- !.
verify_proof(Premises, [H|T], Verified) :-
   rule(Premises, H, Verified),
   append(Verified, [H], VerifiedNew),
   verify_proof(Premises, T, VerifiedNew).

% PREMISE
rule(Premises, [_, Formula, premise], _) :-
   member(Formula, Premises).

% ASSUMPTION
rule(Premises, [[R, Formula, assumption] | T], Verified) :-
   append(Verified, [[R, Formula, assumption]], VerifiedNew),
   verify_proof(Premises, T, VerifiedNew).

%LAW OF EXCLUDED MIDDLE
rule(_, [_, or(X, neg(X)), lem], _Verified).

% COPY
rule(_, [_, X, copy(R)], Verified) :-
   member([R, X, _], Verified).

% DOUBLE NEGATION INTRODUCTION
rule(_, [_, neg(neg(X)), negnegint(R)], Verified) :-
   member([R, X, _], Verified).

% DOUBLE NEGATION ELIMINATION
rule(_, [_, X, negnegel(R)], Verified) :-
   member([R, neg(neg(X)), _], Verified).

% AND INTRODUCTION
rule(_, [_, and(X, Y), andint(R1, R2)], Verified) :-
   member([R1, X, _], Verified),
   member([R2, Y, _], Verified).

% 1st AND ELIMINATION
rule(_, [_, X, andel1(R)], Verified) :-
   member([R, and(X, _), _], Verified).

% 2nd AND ELIMINATION
rule(_, [_, Y, andel2(R)], Verified) :-
   member([R, and(_, Y), _], Verified).

% 1st OR INTRODUCTION
rule(_, [_, or(X, _), orint1(R)], Verified) :-
   member([R, X, _], Verified).

% 2nd OR INTRODUCTION
rule(_, [_, or(_, X), orint2(R)], Verified) :-
   member([R, X, _], Verified).

% OR ELIMINATION (requires assumptions)
rule(_, [_, X, orel(R, S1, S2, T1, T2)], Verified) :-
   member([R, or(A, B), _], Verified),
   
   member([[S1, A, assumption] | TailS], Verified),
   last(TailS, BoxConclusionS),
   [S2, X, _] = BoxConclusionS,

   member([[T1, B, assumption] | TailT], Verified),
   last(TailT, BoxConclusionT),
   [T2, X, _] = BoxConclusionT.

% IMPLICATION INTRODUCTION
rule(_, [_, imp(X, Y), impint(R1, R2)], Verified) :-
   member([[R1, X, assumption] | T], Verified),
   last([[R1, X, assumption] | T], BoxConclusion),
   [R2, Y, _] = BoxConclusion.

% IMPLICATION ELIMINATION
rule(_, [_, Y, impel(R1, R2)], Verified) :-
   member([R1, X, _], Verified),
   member([R2, imp(X, Y), _], Verified).

% NEGATION INTRODUCTION
rule(_, [_, neg(X), negint(R1, R2)], Verified) :-
   member([[R1, X, assumption] | T], Verified),
   last(T, BoxConclusion),
   [R2, cont, _] = BoxConclusion.

% NEGATION ELIMINATION
rule(_, [_, cont, negel(R1, R2)], Verified) :-
   member([R1, X, _], Verified),
   member([R2, neg(X), _], Verified).

% CONTRADICTION ELIMINATION
rule(_, [_, _X, contel(R)], Verified) :-
   member([R, cont, _], Verified).

% MODUS TOLLENS
rule(_, [_, neg(X), mt(R1, R2)], Verified) :-
   member([R1, imp(X, Y), _], Verified),
   member([R2, neg(Y), _], Verified).

% PROOF BY CONTRADICTION
rule(_, [_, X, pbc(R1, R2)], Verified) :-
   member([[R1, neg(X), assumption] | T], Verified),
   last(T, BoxConclusion),
   [R2, cont, _] = BoxConclusion.
\end{lstlisting}

\end{document}

